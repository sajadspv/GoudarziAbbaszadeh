\documentclass{article}

% Language setting
% Replace `english' with e.g. `spanish' to change the document language
\usepackage[english]{babel}
\usepackage{setspace}
\usepackage{tabularx}
\usepackage{hyperref}
\hypersetup{pdfborder=0 0 0}

% Set page size and margins
% Replace `letterpaper' with `a4paper' for UK/EU standard size
\usepackage[a4paper,top=2cm,bottom=2cm,left=3cm,right=3cm,marginparwidth=1.75cm]{geometry}

% Useful packages
\usepackage{amsmath}

\title{Requirements Analysis and Specification Document\\(RASD)}
\author{Mohammad Amin Abbaszadeh 
Sajjad}

\date{\vspace{-5ex}}


\begin{document}
\maketitle
\doublespacing

\tableofcontents
\newpage
\section{Introduction}

\subsection{Purpose}

Nowadays, with a great pace in the usage of fossil fuels by the transportation system, significant negative environmental consequences are rising. A well-know solution to overcome this problem is Electric mobility (e-Mobility). Exploiting  e-Mobility leads to several challenges that can be handled through softwares. When using and electric automobile, it is substantial to have a feasible plan for charging the car in way that pose the least limitation on users consumption. Whit the help of a software system called e-Mobility Service Providers (eMSPs) drivers can proceed with managing their car charging. The eMSPs cooperate with  Charging Point Operators (CPOs) through the so-called Charge Point Management System (CPMS). The CPMS provide electricity to the connected vehicle by obtaining it from an exteranl Distribution System Operators (DSOs). The whole process of tranfering electricity from the source where it is produced to the battery inside an automobile is handled and managed by CPMS and eMSPs. 
The whole process of requirement engineering of these systems are elaborated in this document.

\subsubsection{Goals}
\vspace{0.2cm}

\begin{tabularx}{1\textwidth} { 
  | >{\raggedright\arraybackslash}X 
  | >{\raggedright\arraybackslash}X
  | }


\hline
\textbf {Goals} & \textbf{ Description} \\ 
 \hline
G1  &   User should access to all near charging stations as well as the cost and special offers of the stations \\
\hline
G2  &   User should be able to reserve a charge in a selected station in a an possible time frame\\
\hline
G3 &   User should be able to start the charging process at a specific station \\
\hline
G4 & User should be notified via an application when the charging is finished \\
\hline
G5 & User should be able to pay for the charging service online through the application \\
\hline
G6 & CPOs should be able to spot the location of charging stations and know about their number of charging sockets
available, their type such as slow/fast/rapid, their cost, and, if all sockets of a certain type are occupied, the estimated amount of time until the first socket of that type is freed \\
\hline
G7 & CPOs should be able to start charging a vehicle according to the amount of power supplied by the socket, and monitor the charging process to infer when the battery is full\\
\hline

G8 & CPOs should be able to know the “internal” status of a charging station \\
\hline

G9 & CPOs should be able to retrieve information about the current price of energy from DSOs\\
\hline

G10 & CPOs should be able to decide from which DSO to acquire energy\\
\hline

G11 & CPOs should be able to dynamically decide where to get energy for charging (station battery, DSO, or a mix thereof according to availability and cost)\\

\hline
\end{tabularx}

\subsection{Scope}
Such a system could be developed in a remarkably smart way such that recommend the user a charging plan by accessing to the user calendar, battery status, navigation system, and etc. However, in this document the scope of the project that is elaborated based on the goals mentioned above as well as the fact that an eMSP can interact with multiple CPMSs, each one owned by a different CPO. The interaction approach between eMSP and CPMSs (synchronous, asynchronous, a mixture between the two).

\subsubsection{World Phenomena}
\begin{tabularx}{1\textwidth} { 
  | >{\raggedright\arraybackslash}X 
  | >{\raggedright\arraybackslash}X
  | }
\hline
\textbf {World Phenomena} & \textbf{Description} \\ 
\hline
WP1 & Driver needs to refill the battery of the car whenever it is required \\ 
\hline
WP2 & A driver can spot a charging station nearby (less than 100km) according to the distribution of charging stations geographically \\
\hline
WP3 & The price of electricity fluctuate frequently\\
\hline
WP4 & Electricity is generated in some factories and distributes by DSOs \\
\hline
\end{tabularx}
%%%%%%%%
%%%%%%%%
%%%%%%%%
%%%%%%%%
%%%%%%%%
\subsubsection{Shared Phenomena}
\begin{itemize}
    \item \textbf{controlled by the world and observed by the machine}
\end{itemize}
\begin{tabularx}{1\textwidth} { 
  | >{\raggedright\arraybackslash}X 
  | >{\raggedright\arraybackslash}X
  | }
\hline

\textbf {Shared Phenomena } & \textbf{Description} \\
\hline
SP1 & A user sign up in application of log in if already existed\\
\hline
SP2 & A user can book a charging \\
\hline
SP3 & A user can start the process of charging at a certain station\\
\hline
SP4 & A user pays the amount of charging by the application\\
\hline
SP5 & A CPO manage energy through application manually\\
\hline


\end{tabularx}
%%%%%%%%%%%%%%%%%%%%%%%%%%%%%%%%%%%%%%%%%%%%%%%%%%%%%%%%%%%%%%%%
\begin{itemize}
    \item \textbf{controlled by the machine and observed by the world}
\end{itemize}
\begin{tabularx}{1\textwidth} { 
  | >{\raggedright\arraybackslash}X 
  | >{\raggedright\arraybackslash}X
  | }
\hline

\textbf {Shared Phenomena } & \textbf{Description} \\
\hline
SP & The system shows to the user the nearby stations plus the cost and special offers of the station\\
\hline
SP & The system notifies the user when the charging process is finished\\
\hline



\end{tabularx}


\subsection{Definitions,Acronyms,Abbreviations}

\subsubsection{Definitions}
\begin{tabularx}{1\textwidth} { 
  | >{\raggedright\arraybackslash}X 
  | >{\raggedright\arraybackslash}X
  | }
\hline

CPO & CPO are companies that manage the charging stations.\\
\hline

CPMS & CPMS is an IT infrastructure ( A software system) that enables the CPOs to manage the charging stations either manually or automatically by the CPMS itself \\
\hline

eMSP & eMSP is a sotware system that enables drivers to manage charging their vehicles \\
\hline

Driver & Driver is a user of the system who interacts with the eMSP system to manage charging a vehicle \\
\hline






\end{tabularx}
\subsubsection{Abbreviations}

\vspace{0.2cm}

\begin{tabularx}{1\textwidth} { 
  | >{\raggedright\arraybackslash}X 
  | >{\raggedright\arraybackslash}X
  | }
 \hline
\textbf {Goals} & \textbf{ Description} \\ 
\hline
 RASD  &   Requirements Analysis and Specification Document \\
\hline
CPO & Charging Point Operator \\
\hline
CPMS & Charge Point Management System \\
\hline
eMSP & e-Mobility Service Provider \\
\hline
DSO & Distribution System Operator \\
\hline
WPx & World phenomena number x \\
\hline
SPx & Shared phenomena number x\\
\hline


\end{tabularx}

 
\subsection{Revision History}

\subsection{Reference Documents}

\begin{itemize}
    \large
    \item \href{https://webeep.polimi.it/pluginfile.php/515844/mod_folder/content/0/ProjectToBeReviewed/RASD.pdf}{DREAM\_RASD.pdf}\\
    \item \href{https://webeep.polimi.it/pluginfile.php/515844/mod_folder/content/0/Assignment%20RDD%20AY%202022-2023_v3.pdf?forcedownload=1}{Document of The Project}
\end{itemize}

\subsection{Document Structure}

\section{Overall description}

\subsection{Product perspective}

\subsubsection{Scenarios}

\subsubsection{Class diagram}

\subsubsection{State charts}

\subsection{Product functions}

\subsubsection{X}

\subsection{User characteristic}

\subsection{Assumptions, dependencies and constraints}

\subsubsection{Domain assumptions}

\section{Specific Requirements }

\subsection{External Interface Requirements }

\subsubsection{User Interface}

\subsubsection{User Interface}

\subsubsection{Communication Interfaces}

\subsection{Functional Requirements}

\subsubsection{Mapping on Goals}

\subsubsection{Use Cases}

\subsubsection{Use Case Diagram}

\subsubsection{Mapping on Requirements}

\subsection{Performance Requirements}

\subsection{Design Constraints}

\subsubsection{Standards Compliance}

\subsubsection{Hardware Limitations}

\subsection{Software System Attributes}

\subsubsection{Reliability}

\subsubsection{Availability}

\subsubsection{Security}

\subsubsection{Maintainability}

\subsubsection{Portability}

\section{Formal Analysis}

\subsection{Alloy Code}

\subsubsection{X}

\subsection{XX}

\section{Effort Spent}

\section{References }








% \begin{table}
% \centering
% \begin{tabular}{l|r}
% Item & Quantity \\\hline
% Widgets & 42 \\
% Gadgets & 13
% \end{tabular}
% \caption{\label{tab:widgets}An example table.}
% \end{table}


\end{document}